Guilherme e Gabriela são dois irmãos apaixonados por arte. Desde pequenos, sempre competiram para ver quem conseguia criar o melhor desenho. Desta vez, eles decidiram transformar o muro de casa em uma obra de arte colorida, utilizando graffiti. Cada um tem uma cor favorita: Gabriela gosta de laranja, enquanto Guilherme prefere o roxo. No entanto, essa simples discordância transformou a atividade em uma competição. Como em qualquer disputa entre irmãos, o objetivo não é apenas grafitar, mas ver quem consegue cobrir a maior área do muro com sua cor favorita.

O muro possui uma altura fixa de 1 metro e inicialmente está em branco. Gabriela e Guilherme se revezam para grafitar. Como Gabriela é mais velha, ela sempre começa. A cada turno, eles escolhem uma posição e uma distância horizontal no muro, e toda a área dentro dessa distância é grafitada.

Como ambos são extremamente competitivos, eles podem grafitar por cima das cores já preenchidas pelo outro, e a cor mais recente prevalece em cada posição.

\subsection*{Entrada}

A primeira linha contém dois inteiros \textit{n} e \textit{m} ($1 \leq \textit{n} \leq 10^5, 1 \leq \textit{m} \leq 10^5$), representando o comprimento do muro em metros e o número de turnos, respectivamente.

As próximas \textit{m} linhas descrevem as ações dos irmãos em cada turno.

Cada uma dessas linhas contém dois inteiros \textit{$p_i$} e \textit{$d_i$} ($0 \leq \textit{$p_i$} < \textit{n}, 1 \leq \textit{$d_i$} \leq n - p_i$), onde \textit{$p_i$} é a posição inicial,  começando de 0, e \textit{d} é a distância grafitada a partir dessa posição.

\subsection*{Saída}

Imprima o nome do irmão que grafitou a maior área da parede. Se ambos grafitaram a mesma área, imprima "Empate".

\newpage

%----- Exemplo 1 -----%
\begin{table}[!h]
\centering
\begin{tabular}{|l|l|}
\hline
\begin{minipage}[t]{3in}
\textbf{Exemplo de entrada}
\begin{verbatim}
8 4
0 4
4 4
0 8
3 3
\end{verbatim}
\vspace{1mm}
\end{minipage}
&
\begin{minipage}[t]{3in}
\textbf{Exemplo de saída}
\begin{verbatim}
Gabriela
\end{verbatim}
\vspace{1mm}
\end{minipage} \\
\hline
\end{tabular}
\end{table}

%----- Exemplo 2 -----%
\begin{table}[!h]
\centering
\begin{tabular}{|l|l|}
\hline
\begin{minipage}[t]{3in}
\textbf{Exemplo de entrada}
\begin{verbatim}
10 7
0 10
0 8
2 7
3 4
4 2
8 2
5 1
\end{verbatim}
\vspace{1mm}
\end{minipage}
&
\begin{minipage}[t]{3in}
\textbf{Exemplo de saída}
\begin{verbatim}
Guilherme
\end{verbatim}
\vspace{1mm}
\end{minipage} \\
\hline
\end{tabular}
\end{table}

%----- Exemplo 3 -----%
\begin{table}[!h]
\centering
\begin{tabular}{|l|l|}
\hline
\begin{minipage}[t]{3in}
\textbf{Exemplo de entrada}
\begin{verbatim}
12 8
0 1
1 11
5 6
7 3
2 2
0 5
0 8
2 3
\end{verbatim}
\vspace{1mm}
\end{minipage}
&
\begin{minipage}[t]{3in}
\textbf{Exemplo de saída}
\begin{verbatim}
Empate
\end{verbatim}
\vspace{1mm}
\end{minipage} \\
\hline
\end{tabular}
\end{table}
