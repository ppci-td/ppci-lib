Às vezes, é preciso ter aquela conversa difícil e dizer o que é necessário. É ou não é?

Yara e sua colega Karen compartilham o mesmo apartamento. A papagaia da Karen, Lori, fica o dia todo presa na gaiola, sem ninguém para conversar, pois Karen estuda o dia inteiro e só chega de noite para dar atenção à coitada. Yara tem dó da passarinha, e precisa mostrar para Karen como a passarinha se sente abandonada e que é preciso tomar uma providência.

Yara é estudante de biologia, e por observação chegou à hipótese de que papagaios falam muito quando estressados, e às vezes também podem falar ao contrário, de trás para frente. Então, ela gravou a papagaia ao longo de um dia e obteve uma string $S$, que contém letras correspondentes a sons, os quais podem ter sido falas da Lori ou ruídos de fundo. A bichana sempre fala a mesma palavra $P$, podendo ser de frente para trás ou de trás para frente. 

É preciso que você conte em quantos momentos, ao longo da gravação, houve um som que possa ter sido a Lori falando. Dessa forma, Yara poderá estudar o caso, e quem sabe mostrar para Karen como o comportamento da Lori é consequência da tristeza e estresse de estar sozinha e sem interação o dia inteiro.


\subsection*{Entrada}
A primeira linha contém uma string $S$, de tamanho $|S|$ ($1\leq |S|\leq 2\cdot 10^5$).

A segunda linha contém uma string $P$, de tamanho $|P|$ ($1\leq |P|\leq 2\cdot 10^5$).

Ambas strings são formadas por caracteres minúsculos do alfabeto inglês (i.e. de \texttt{`a'} a \texttt{`z'}).


\subsection*{Saída}
Imprima a quantidade de momentos distintos da gravação em que houve um som que possa ser a papagaia falando.


%----- Exemplo 1 -----%
\newpage
\begin{table}[!h]
\centering
\begin{tabular}{|l|l|}
\hline
\begin{minipage}[t]{3in}
\textbf{Exemplo de entrada}
\begin{verbatim}
zzzgrrauauzzz
uauarrg
\end{verbatim}
\vspace{1mm}
\end{minipage}
&
\begin{minipage}[t]{3in}
\textbf{Exemplo de saída}
\begin{verbatim}
1
\end{verbatim}
\vspace{1mm}
\end{minipage} \\
\hline
\end{tabular}
\end{table}

%----- Exemplo 2 -----%
\begin{table}[!h]
\centering
\begin{tabular}{|l|l|}
\hline
\begin{minipage}[t]{3in}
\textbf{Exemplo de entrada}
\begin{verbatim}
orolorolouolorolol
loro
\end{verbatim}
\vspace{1mm}
\end{minipage}
&
\begin{minipage}[t]{3in}
\textbf{Exemplo de saída}
\begin{verbatim}
5
\end{verbatim}
\vspace{1mm}
\end{minipage} \\
\hline
\end{tabular}
\end{table}


%----- Exemplo 3 -----%
\begin{table}[!h]
\centering
\begin{tabular}{|l|l|}
\hline
\begin{minipage}[t]{3in}
\textbf{Exemplo de entrada}
\begin{verbatim}
arararara
arara
\end{verbatim}
\vspace{1mm}
\end{minipage}
&
\begin{minipage}[t]{3in}
\textbf{Exemplo de saída}
\begin{verbatim}
3
\end{verbatim}
\vspace{1mm}
\end{minipage} \\
\hline
\end{tabular}
\end{table}
