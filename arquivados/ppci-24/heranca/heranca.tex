Você acaba de receber uma notícia inesperada: seu tio-avô, morador de Nlogônia, faleceu e deixou uma herança especial para você. Porém, sabendo da sua paixão por problemas de lógica e desafios matemáticos, ele deixou um enigma como parte de sua herança.

Na casa dele, você encontra uma mochila mágica com um limite de peso $W$ que pode ser carregado dentro dela. Junto à mochila, há uma carta explicando que ela pode ser preenchida com uma série de itens valiosos, cada um com um peso e um valor específico. No entanto, você não precisa pegar os itens inteiros – pode escolher levar apenas uma fração de cada item, se assim desejar.

Seu desafio é simples: escrever um algoritmo que defina o valor máximo dos itens para o peso $w$ de sua mochila.

\subsection*{Entrada}

A primeira linha contém 2 inteiros $N$ ($1 \leq N \leq 10^5$) e $W$ ($1 \leq W \leq 10^3$) que indicam, respectivamente, o número de itens disponíveis e o peso máximo suportado pela mochila.

As próximas $N$ linhas contêm 2 inteiros $P$ ($1 \leq P \leq 10^6$) e $V$ ($1 \leq V \leq 10^8$) indicando o peso e o valor de cada item.

\subsection*{Saída}

Imprima um número com 2 casas após a vírgula, indicando o valor máximo.

Utilize números com precisão dupla.

%----- Exemplo 1 -----%
\begin{table}[!h]
\centering
\begin{tabular}{|l|l|}
\hline
\begin{minipage}[t]{3in}
\textbf{Exemplo de entrada}
\begin{verbatim}
3 50
10 60
20 100
30 120
\end{verbatim}
\vspace{1mm}
\end{minipage}
&
\begin{minipage}[t]{3in}
\textbf{Exemplo de saída}
\begin{verbatim}
240.00
\end{verbatim}
\vspace{1mm}
\end{minipage} \\
\hline
\end{tabular}
\end{table}

%----- Exemplo 2 -----%
\begin{table}[!h]
\centering
\begin{tabular}{|l|l|}
\hline
\begin{minipage}[t]{3in}
\textbf{Exemplo de entrada}
\begin{verbatim}
2 50
20 10
50 3
\end{verbatim}
\vspace{1mm}
\end{minipage}
&
\begin{minipage}[t]{3in}
\textbf{Exemplo de saída}
\begin{verbatim}
11.80
\end{verbatim}
\vspace{1mm}
\end{minipage} \\
\hline
\end{tabular}
\end{table}

