O futebol é o esporte mais popular do mundo, com entusiastas espalhados globalmente. Rafael, além de ser um grande fã do esporte, também é analista de desempenho de um clube de futebol profissional chamado Universitários FC.

O técnico do time solicitou que Rafael desenvolvesse um programa para ajudar o time a melhorar suas chances de marcar gols. Contudo, Rafael não é especialista em programação, e por isso ele decidiu pedir a sua ajuda para resolver este problema.

A sua tarefa é criar um programa que analise se é possível que o Universitários FC troque passes de maneira eficiente, de modo que a bola saia dos pés do goleiro e chegue até o centroavante, o jogador responsável por finalizar as jogadas.

A entrada do programa consiste nas possíveis opções de passes entre os jogadores. Cada passe indica que a bola pode ser passada de um jogador X para um jogador Y (e vice-versa). Note que as opções de passes podem ser limitadas, o programa deve encontrar o tamanho do menor caminho possível que a bola fará entre os jogadores. Em outras palavras, o programa deve mostrar a quantidade mínima de jogadores que participarão na sequência de passes, de forma que o número total de jogadores envolvidos seja o menor possível.

\subsection*{Entrada}

A primeira linha contém um número inteiro $N$ ($1 \leq N \leq 55$), representando o número de passes possíveis.

As próximas $N$ linhas descrevem as opções de passe, onde a bola pode ser passada do jogador X para o jogador Y e vice-versa. Os jogadores são numerados de 1 a 11, sendo que o jogador 1 é o goleiro e o jogador 9 é o centroavante.

\subsection*{Saída}

Imprima a quantidade mínima de jogadores que participarão da troca de passes, incluindo o goleiro (jogador 1) e o centroavante (jogador 9).

Caso não seja possível completar a sequência de passes até o centroavante, imprima $-1$.

%----- Exemplo 1 -----%
\begin{table}[!h]
\centering
\begin{tabular}{|l|l|}
\hline
\begin{minipage}[t]{3in}
\textbf{Exemplo de entrada}
\begin{verbatim}
11
1 3
1 4
3 2
4 6
3 5
5 10
5 8
10 8
2 7
10 11
11 9
\end{verbatim}
\vspace{1mm}
\end{minipage}
&
\begin{minipage}[t]{3in}
\textbf{Exemplo de saída}
\begin{verbatim}
6
\end{verbatim}
\vspace{1mm}
\end{minipage} \\
\hline
\end{tabular}
\end{table}

%----- Exemplo 2 -----%
\begin{table}[!h]
\centering
\begin{tabular}{|l|l|}
\hline
\begin{minipage}[t]{3in}
\textbf{Exemplo de entrada}
\begin{verbatim}
9
1 3
1 4
3 2
4 6
4 5
5 10
5 8
11 9
9 7
\end{verbatim}
\vspace{1mm}
\end{minipage}
&
\begin{minipage}[t]{3in}
\textbf{Exemplo de saída}
\begin{verbatim}
-1
\end{verbatim}
\vspace{1mm}
\end{minipage} \\
\hline
\end{tabular}
\end{table}

%----- Exemplo 3 -----%
\begin{table}[!h]
\centering
\begin{tabular}{|l|l|}
\hline
\begin{minipage}[t]{3in}
\textbf{Exemplo de entrada}
\begin{verbatim}
13
1 3
1 4
3 2
4 6
3 5
4 5
5 10
5 8
10 11
8 7
1 9
11 9
7 9
\end{verbatim}
\vspace{1mm}
\end{minipage}
&
\begin{minipage}[t]{3in}
\textbf{Exemplo de saída}
\begin{verbatim}
2
\end{verbatim}
\vspace{1mm}
\end{minipage} \\
\hline
\end{tabular}
\end{table}
