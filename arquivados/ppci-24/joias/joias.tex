Você é um explorador em uma perigosa expedição nas selvas de Zathura, uma terra lendária conhecida por esconder um misterioso tesouro. 
Diz a lenda que esse tesouro é protegido por enigmas e armadilhas que só podem ser vencidos por aqueles que sabem o caminho mais curto até ele.

Determine o número mínimo de movimentos que você precisa para sair de sua posição inicial \((sx, sy)\) e alcançar a posição do tesouro \((tx, ty)\). 
Lembre-se de que a selva é cheia de desafios, e você só pode mover-se nas quatro direções: para cima, para baixo, para a esquerda e para a direita. 
Não é possível atravessar as bordas da selva.

\subsection*{Entrada}
A primeira linha contém um inteiro \(N\) \((1 \leq N \leq 1000)\), representando o tamanho do tabuleiro \(N \times N\).

A segunda linha contém dois inteiros \(sx\) e \(sy\) \((0 \leq sx, sy < N)\), que representam as coordenadas de sua posição inicial no tabuleiro.

A terceira linha contém dois inteiros \(tx\) e \(ty\) \((0 \leq tx, ty < N)\), que representam as coordenadas do tesouro.

\subsection*{Saída}

Imprima o número mínimo de movimentos que o explorador precisa fazer para alcançar o tesouro.

\subsection*{Exemplo de Entrada e Saída}

%----- Exemplo 1 -----%  
\begin{table}[!h]  
\centering  
\begin{tabular}{|l|l|}  
\hline  
\begin{minipage}[t]{3in}  
\textbf{Exemplo de entrada}  
\begin{verbatim}  
5  
0 0  
3 3  
\end{verbatim}  
\vspace{1mm}
\end{minipage}  
&  
\begin{minipage}[t]{3in}  
\textbf{Exemplo de saída}  
\begin{verbatim}  
6  
\end{verbatim}  
\vspace{1mm}
\end{minipage} \\  
\hline  
\end{tabular}  
\end{table}

%----- Exemplo 2 -----%  
\begin{table}[!h]  
\centering  
\begin{tabular}{|l|l|}  
\hline  
\begin{minipage}[t]{3in}  
\textbf{Exemplo de entrada}  
\begin{verbatim}  
7  
1 1  
5 5  
\end{verbatim}  
\vspace{1mm}
\end{minipage}  
&  
\begin{minipage}[t]{3in}  
\textbf{Exemplo de saída}  
\begin{verbatim}  
8  
\end{verbatim}  
\vspace{1mm}
\end{minipage} \\  
\hline  
\end{tabular}  
\end{table}
