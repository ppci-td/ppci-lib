Kleber está participando da sua primeira convenção de colecionadores de números. Nesta convenção cada colecionador pode doar ou receber números de outros colecionadores, alterando assim sua coleção.

Há um estande de confirmação de coleção que dá uma insígnia ao colecionador se o $\oplus^\textbf{*}$(operação binária xor) de algum dos $E$ números distintos presentes no estande com TODOS os $N$ números da coleção atual de um colecionador resultarem em um número cuja representação binária contém apenas bits 1, desconsiderando zeros a esquerda. O estande dará apenas uma insígnia por coleção, independente de quantos números do estande atendam o requisito.

Kleber fará e/ou receberá $Q$ doações de números durante a convenção e deseja saber qual o número máximo de insígnias que ele consegue obter. Ele pode ir ao estande para confirmar sua coleção quantas vezes quiser, essa verificação pode ser feita sempre que a coleção for alterada.

Veja a explicação do exemplo teste:

Suponha que Kleber tem uma coleção: $1, 2, 8$ e que o estande de confirmação tem os seguintes números: $4, 5, 7$.

Durante a convenção aconteceram as seguintes doações:

$+$ $3$ deixando sua coleção como: $1, 2, 3, 8$.

$+$ $4$ deixando sua coleção como: $1, 2, 3, 4, 8$.

$-$ $8$ deixando sua coleção como: $1, 2, 3, 4$.

$-$ $1$ deixando sua coleção como: $2, 3, 4$.

($+$ simbolizam que Kleber recebeu uma doação e $-$ simboliza que Kleber fez uma doação).

Durante esta convenção Kleber conseguiu 2 insígnias:

$1\oplus2\oplus8\oplus\textbf{4} = 15_{10} = 1111_2$.

$1\oplus2\oplus3\oplus8\oplus\textbf{7} = 15_{10} = 1111_2$.

\small{\textbf{*} $\oplus$ (xor) é a operação binária na qual: $1 \oplus 0 = 1$, $0 \oplus 1 = 1$, $0 \oplus 0 = 0$ e $1 \oplus 1 = 0$}. 

\subsection*{Entrada}
A primeira linha contém três inteiros $N$ e $E$ ($1 \leq N,E \leq 2 \cdot 10^5$), quantos números Kleber tem em sua coleção, quantos números distintos o estande de confirmação de coleção possui, respectivamente.

A segunda linha contém $N$ números $a_i$ ($0 \leq a_i \leq 10^9$), a coleção de Kleber.

A terceira linha contém $E$ números distintos do estande de confirmação de coleção $a_j$ ($0 \leq a_j \leq 10^9$).

A quarta linha contém um inteiro $Q$ ($1 \leq Q \leq 2 \cdot 10^5$), quantas doações serão feitas e/ou recebidas por Kleber

As próximas $Q$ linhas contém operações $-$ e/ou $+$ de números $a_k$ ($0 \leq a_k \leq 10^9$), a descrição das doações que aconteceram durante a convenção.

\subsection*{Saída}
Imprima o números máximo de insígnias que Kleber conseguirá receber.

\newpage
\begin{table}[!h]
\centering
\begin{tabular}{|l|l|}
\hline
\begin{minipage}[t]{3in}
\textbf{Exemplo de entrada}
\begin{verbatim}
3 3
1 2 8
4 5 7
4
+ 3
+ 4
- 8
- 1
\end{verbatim}
\vspace{1mm}
\end{minipage}
&
\begin{minipage}[t]{3in}
\textbf{Exemplo de saída}
\begin{verbatim}
2
\end{verbatim}
\vspace{1mm}
\end{minipage} \\
\hline
\end{tabular}
\end{table}
