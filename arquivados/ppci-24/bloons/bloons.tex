No jogo \textbf{Bloons Tower Defense}, existem macacos que defendem o território deles, que é constantemente atacado por uma sequência de balões com diferentes camadas de resistência. Cada macaco pode arremessar dardos que dão dano e destroem um número específico de camadas de um balão. Cada balão, por sua vez, tem um número de camadas a serem destruídas. Um balão só é totalmente destruído quando o último dardo que o acertar der dano \textbf{exatamente} igual à quantidade de camadas restantes do balão.

Dada uma sequência de $N$ balões, onde cada balão tem $h_i$ camadas (ou seja, o balão $i$ tem resistência $h_i$), temos disponíveis $K$ tipos de macacos, onde o macaco do tipo $j$ pode arremessar um dardo que destrói exatamente $d_j$ camadas de um balão.

Você deve determinar o número mínimo de dardos necessários para destruir completamente todos os balões da sequência. Se não for possível destruir todos os balões, retorne -1. E lembrando que um balão é totalmente destruído somente se os dano causado nele é \textbf{exatamente} o número de camadas dele. E ressaltando também que o mesmo macaco pode arremessar dardos quantas vezes quiser em qualquer balão.

\subsection*{Entrada}
    A primeira linha contém dois inteiros $N$ $(1 \leq N \leq 1000)$ e $K$ $(1 \leq K \leq 100)$, representando o número de balões e o número de macacos, respectivamente.

    A segunda linha contém $N$ inteiros, onde o $i$-ésimo inteiro representa $h_i$ $(1 \leq h_i \leq 1000)$, o número de camadas de resistência do $i$-ésimo balão.

    A terceira linha contém $K$ inteiros, onde o $j$-ésimo inteiro representa $d_j$ $(1 \leq d_j \leq 1000)$, o número de camadas que o dardo arremessado pelo $j$-ésimo macaco consegue destruir.

\subsection*{Saída}
Imprima o número mínimo de dardos necessários para destruir todos os balões. Se não for possível destruir todos os balões, imprima -1.

%----- Exemplo 1 -----%
\newpage
\begin{table}[!h]
\centering
\begin{tabular}{|l|l|}
\hline
\begin{minipage}[t]{3in}
\textbf{Exemplo de entrada}
\begin{verbatim}
4 3
5 10 6 8
2 3 5
\end{verbatim}
\vspace{1mm}
\end{minipage}
&
\begin{minipage}[t]{3in}
\textbf{Exemplo de saída}
\begin{verbatim}
7
\end{verbatim}
\vspace{1mm}
\end{minipage} \\
\hline
\end{tabular}
\end{table}

%----- Exemplo 2 -----%
\begin{table}[!h]
\centering
\begin{tabular}{|l|l|}
\hline
\begin{minipage}[t]{3in}
\textbf{Exemplo de entrada}
\begin{verbatim}
2 2
7 11
2 5
\end{verbatim}
\vspace{1mm}
\end{minipage}
&
\begin{minipage}[t]{3in}
\textbf{Exemplo de saída}
\begin{verbatim}
6
\end{verbatim}
\vspace{1mm}
\end{minipage} \\
\hline
\end{tabular}
\end{table}


%----- Exemplo 3 -----%
\begin{table}[!h]
\centering
\begin{tabular}{|l|l|}
\hline
\begin{minipage}[t]{3in}
\textbf{Exemplo de entrada}
\begin{verbatim}
2 2
7 8
3 5
\end{verbatim}
\vspace{1mm}
\end{minipage}
&
\begin{minipage}[t]{3in}
\textbf{Exemplo de saída}
\begin{verbatim}
-1
\end{verbatim}
\vspace{1mm}
\end{minipage} \\
\hline
\end{tabular}
\end{table}
