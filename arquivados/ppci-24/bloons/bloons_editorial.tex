Para resolver o problema, pode-se utilizar programação dinâmica, considerando cada balão de forma independente. Assim, para cada balão, queremos encontrar a quantidade mínima de dardos para destruí-lo, usando os dados disponíveis. Vamos definir:

\[
\begin{align*}
dp[x]: &\ \text{o número mínimo de dardos necessários para destruir um balão com} \\
       &\ \text{resistência } x.
\end{align*}
\]

\[
dp[0] = 0,
\]
já que para um balão com 0 camadas de resistência, não precisamos de nenhum dardo.

Para valores maiores de resistência, definimos \(dp[x] = \infty\), representando que ainda não sabemos como destruir aquele balão.

O estado de transição será:
Para cada camada \(x\) e cada macaco distinto com dano \(d_j\), se for possível usar o macaco \((x \geq d_j)\), então podemos atualizar o número mínimo de dardos:

\[
dp[x] = \min(dp[x], dp[x - d_j] + 1)
\]

Essa transição verifica se podemos destruir o balão com resistência \(x\) usando um dardo que destrói \(d_j\) camadas e o número mínimo de dardos necessários para destruir um balão de resistência \(x - d_j\).

Assim, para cada dardo \(d_j\) disponível, atualizamos o array \(dp\) para cada resistência de balão possível \(dp[x]\). Se, ao final do processo, \(dp[h_i] = \infty\), isso significa que não é possível destruir o balão de resistência \(h_i\) com os dardos disponíveis, então retornamos -1. Caso contrário, somamos o total de dardos mínimos necessários para destruir todos os balões.


\newpage

\section*{Código}

\begin{lstlisting}[language=C++]
#include <bits/stdc++.h>
#define ll long long
using namespace std;
const ll INF = INT_MAX;

int main() {
    ll N, K;
    ll MaximoDeCamadas = 1000;
    
    cin >> N >> K;
    
    vector<ll> baloes(N);
    vector<ll> macacos(K);

    for (ll i = 0; i < N; ++i) {
        cin >> baloes[i];
    }

    for (ll i = 0; i < K; ++i) {
        cin >> macacos[i];
    }
    
    vector<ll> dp(MaximoDeCamadas + 1, INF);
    dp[0] = 0;

    for (ll h = 1; h <= MaximoDeCamadas; ++h) {
        for (ll d : macacos) {
            if (h >= d && dp[h - d] != INF) {
                dp[h] = min(dp[h], dp[h - d] + 1);
            }
        }
    }

    ll total_macacos = 0;
    for (ll h : baloes) {
        if (dp[h] == INF) {
            cout << -1 << endl;
            return 0;
        }
        total_macacos += dp[h];
    }
    
    cout << total_macacos << endl;
    return 0;
}
\end{lstlisting}

Complexidade O(N+K)
