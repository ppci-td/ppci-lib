Carlinhos é um grande colecionador de balões e, para proteger sua preciosa coleção, decidiu trancá-la em um cofre com uma senha \textit{s}, composta por \textit{n} números inteiros.
Porém, a memória de Carlinhos não é a das melhores, então ele decidiu escrever a senha em um papel e guardá-la em um lugar seguro.
No entanto, para garantir que ninguém descubra a senha, Carlinhos resolveu anotá-la de maneira camuflada, seguindo uma regra especial:

\begin{itemize}
    \item A maior sequência de números distintos, ou seja, a maior sequência contínua de números onde nenhum número aparece mais de uma vez, foi invertida.
    \item Caso haja mais de uma sequência de maior tamanho, apenas a primeira, da esquerda para a direita, será invertida.
    \item O restante dos números foi mantido na mesma ordem original.
\end{itemize}

Agora, anos depois, Carlinhos quer pegar seu balão vermelho favorito, mas sua memória falhou novamente. 
Ele não lembra qual sequência havia sido invertida, e tudo o que tem é a versão modificada da senha que ele escreveu no papel.

A sua tarefa é ajudar Carlinhos a pegar o seu balão favorito, descobrindo a senha correta do cofre.

\subsection*{Entrada}

A primeira linha contém um inteiro $n$ ($1 \leq n \leq 10^5$), que representa a quantidade de números da senha.

A segunda linha contém $n$ inteiros $s_i$ ($0 \leq s_i \leq 10^9$), que representam os dígitos da senha como foram escritos no papel.

\subsection*{Saída}

Imprima a senha correta para o cofre, separando os números por espaços. Não imprima o espaço após o último número.

\newpage
%----- Exemplo 1 -----%
\begin{table}[!h]
\centering
\begin{tabular}{|l|l|}
\hline
\begin{minipage}[t]{3in}
\textbf{Exemplo de entrada}
\begin{verbatim}
7
1 1 2 3 4 5 5
\end{verbatim}
\vspace{1mm}
\end{minipage}
&
\begin{minipage}[t]{3in}
\textbf{Exemplo de saída}
\begin{verbatim}
1 5 4 3 2 1 5
\end{verbatim}
\vspace{1mm}
\end{minipage} \\
\hline
\end{tabular}
\end{table}

%----- Exemplo 2 -----%
\begin{table}[!h]
\centering
\begin{tabular}{|l|l|}
\hline
\begin{minipage}[t]{3in}
\textbf{Exemplo de entrada}
\begin{verbatim}
9
1 2 1 3 4 7 3 5 7
\end{verbatim}
\vspace{1mm}
\end{minipage}
&
\begin{minipage}[t]{3in}
\textbf{Exemplo de saída}
\begin{verbatim}
1 7 4 3 1 2 3 5 7
\end{verbatim}
\vspace{1mm}
\end{minipage} \\
\hline
\end{tabular}
\end{table}

%----- Exemplo 3 -----%
\begin{table}[!h]
\centering
\begin{tabular}{|l|l|}
\hline
\begin{minipage}[t]{3in}
\textbf{Exemplo de entrada}
\begin{verbatim}
6
42 42 43 43 44 44
\end{verbatim}
\vspace{1mm}
\end{minipage}
&
\begin{minipage}[t]{3in}
\textbf{Exemplo de saída}
\begin{verbatim}
42 43 42 43 44 44
\end{verbatim}
\vspace{1mm}
\end{minipage} \\
\hline
\end{tabular}
\end{table}
