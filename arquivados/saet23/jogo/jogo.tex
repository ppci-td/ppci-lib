    Carlos e Pedro gostam muito de futebol e querem jogar uma partida um contra o outro. Como ambos são goleiros, eles precisam de jogadores no ataque e na defesa para completar o seu time.

    Para isso, eles têm uma lista de jogadores interessados em participar da partida, contendo os
pontos de ataque e de defesa de todos. Um jogador com pontos de defesa maior que
os pontos de ataque é um defensor; um jogador com pontos de ataque maiores ou
iguais aos de defesa é um atacante; o \textit{talento} de um jogador é dado pela
soma de seus pontos de ataque com seus pontos de defesa.


    Para a divisão ser justa, a escolha dos jogadores será feitas em rodadas. Em
    cada rodada, Carlos e Pedro escolherão um jogador cada para seus times.
Na primeira rodada Carlos escolhe um jogador primeiro; na segunda rodada Pedro
escolhe um jogador primeiro; e assim por diante. Caso o número de jogadores
seja ímpar, um deles não será escolhido.

    A escolha de um jogador é feita pelos seguintes critérios:
\begin{itemize}
    \item Se um time já tiver mais atacantes do que defensores, será escolhido
    para o time o defensor mais talentoso ainda não escolhido. Caso não haja
    mais defensores disponíveis, será escolhido o atacante mais talentoso;
    \item Se um time já tiver mais defensores do que atacantes, será escolhido
    para o time o atacante mais talentoso ainda não escolhido. Caso não haja
    mais atacantes disponíveis, será escolhido o defensor mais talentoso;
    \item Se um time tiver a mesma quantidade de atacantes e defensores, será
    escolhido o jogador mais talentoso ainda não escolhido. Se houver mais de um
    jogador ainda não escolhido com o maior talento, será escolhido um atacante.
\end{itemize}

Sua tarefa é determinar o talento médio do time mais talentoso.

\subsection*{Entrada}

A primeira linha da entrada contém o número $N$ de jogadores a serem escolhidos ($2\leq N\leq 10^5$).
Cada uma das $N$ linhas seguintes contêm 2 inteiros: $K_i$ e $M_i$ ($0\leq
        K_i\leq 99, 0\leq M_i\leq 99$), sendo, respectivamente, os pontos de ataque
e de defesa do jogador $i$.

\subsection*{Saída}

A saída deverá conter uma linha com o talento médio do time mais talentoso, com
exatamente duas casas decimais.

%----- Exemplo 1 -----%
\newpage
\begin{table}[!h]
\centering
\begin{tabular}{|l|l|}
\hline
\begin{minipage}[t]{3in}
\textbf{Exemplo de entrada}
\begin{verbatim}
4
10 20
30 10
10 50
10 5
\end{verbatim}
\vspace{1mm}
\end{minipage}
&
\begin{minipage}[t]{3in}
\textbf{Exemplo de saída}
\begin{verbatim}
37.50
\end{verbatim}
\vspace{1mm}
\end{minipage} \\
\hline
\end{tabular}
\end{table}

%----- Exemplo 2 -----%
\begin{table}[!h]
\centering
\begin{tabular}{|l|l|}
\hline
\begin{minipage}[t]{3in}
\textbf{Exemplo de entrada}
\begin{verbatim}
6
62 1
79 65
71 71
 8 91
71 99
20 24
\end{verbatim}
\vspace{1mm}
\end{minipage}
&
\begin{minipage}[t]{3in}
\textbf{Exemplo de saída}
\begin{verbatim}
125.00
\end{verbatim}
\vspace{1mm}
\end{minipage} \\
\hline
\end{tabular}
\end{table}

%----- Exemplo 3 -----%
\begin{table}[!h]
\centering
\begin{tabular}{|l|l|}
\hline
\begin{minipage}[t]{3in}
\textbf{Exemplo de entrada}
\begin{verbatim}
10
61 55
49 81
96 23
 2 59
56 41
33 20
93 16
72 20
65 66
58 94
\end{verbatim}
\vspace{1mm}
\end{minipage}
&
\begin{minipage}[t]{3in}
\textbf{Exemplo de saída}
\begin{verbatim}
117.40
\end{verbatim}
\vspace{1mm}
\end{minipage} \\
\hline
\end{tabular}
\end{table}
