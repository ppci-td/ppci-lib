Kubitschek e Chekskubit decidiram dar uma pausa em suas atividades criminosas para realizar um duelo no jogo Geoguessr. Neste jogo, cada jogador recebe uma imagem de uma localização aleatória do mundo e deve adivinhar onde essa localização se encontra baseando-se em elementos da imagem (placas, estradas, pessoas, vegetação, arquitetura, bandeiras, etc). Após deduzir a localização do local da imagem, o jogador marca um ponto no mapa mundial representando seu palpite. A pontuação é calculada com base na distância em relação à localização real, que é sempre na coordenada \((0, 0)\).

Cada jogador começa com 5000 pontos de vida. A mecânica do jogo é a seguinte, toda partida começa no Round 1 e vai até o Round determinado previamente, a pontuação depois de cada Round é dada da seguinte forma:
\begin{itemize}
    \item Rounds ímpares: Round de dano. A diferença da distância do jogador mais distante do ponto real \((0, 0)\) para o jogador mais próximo é subtraída do total de pontos de vida do jogador mais distante.
    \item Rounds pares: Round de cura. A diferença da distância do jogador mais distante do ponto real \((0, 0)\) para o jogador mais próximo é adicionada ao total de pontos de vida do jogador mais próximo, sem exceder o limite de 5000 pontos de vida.
\end{itemize}

O jogo termina quando um jogador chega a 0 pontos de vida ou quando ambos os jogadores completam o número de rounds definidos. Se, no último round, ambos os jogadores tiverem o mesmo total de pontos de vida, o resultado é um empate. Caso contrário, o jogador com mais pontos de vida é o vencedor.

\subsection*{Entrada}

\begin{itemize}
\item A primeira linha contém um número inteiro \(N\) \((1 \leq N \leq 100)\), representando o número de rounds. 
\item Em cada uma das próximas \(N\) linhas contém quatro números inteiros \(x_1, y_1, x_2, y_2\) (representando as coordenadas dos palpites dos jogadores em relação ao ponto real \((0, 0)\)), onde \(-10000 \leq x_1, y_1, x_2, y_2 \leq 10000\). Aqui, \((x_1, y_1)\) são as coordenadas do palpite de Kubitschek, e \((x_2, y_2)\) são as coordenadas do palpite de Chekskubit.
\end{itemize}

\subsection*{Saída}

Imprima uma linha contendo  um dos três resultados possíveis da partida:
\begin{itemize}
    \item ``Kubitschek Venceu'' 
    \item ``Chekskubit Venceu'' 
    \item ``Empate'' 
\end{itemize}

\newpage
\begin{table}[!h]
\centering
\begin{tabular}{|l|l|}
\hline
\begin{minipage}[t]{2.5in}
\textbf{Exemplo de entrada}
\begin{verbatim}
2
1 1 3 3
4 4 6 6
\end{verbatim}
\vspace{1mm}
\end{minipage}
&
\begin{minipage}[t]{2.5in}
\textbf{Exemplo de saída}
\begin{verbatim}
Kubitschek Venceu
\end{verbatim}
\vspace{1mm}
\end{minipage} \\
\hline
\end{tabular}
\end{table}

\begin{table}[!h]
\centering
\begin{tabular}{|l|l|}
\hline
\begin{minipage}[t]{2.5in}
\textbf{Exemplo de entrada}
\begin{verbatim}
4
3 3 4 4
5 5 6 6
7 7 8 8
9 9 10 10
\end{verbatim}
\vspace{1mm}
\end{minipage}
&
\begin{minipage}[t]{2.5in}
\textbf{Exemplo de saída}
\begin{verbatim}
Empate
\end{verbatim}
\vspace{1mm}
\end{minipage} \\
\hline
\end{tabular}
\end{table}

\begin{table}[!h]
\centering
\begin{tabular}{|l|l|}
\hline
\begin{minipage}[t]{2.5in}
\textbf{Exemplo de entrada}
\begin{verbatim}
3
10000 10000 0 0
10000 10000 0 0
10000 10000 0 0
\end{verbatim}
\vspace{1mm}
\end{minipage}
&
\begin{minipage}[t]{2.5in}
\textbf{Exemplo de saída}
\begin{verbatim}
Chekskubit Venceu
\end{verbatim}
\vspace{1mm}
\end{minipage} \\
\hline
\end{tabular}
\end{table}
