\documentclass[12pt,oneside]{article} % Uma Coluna e lingua portuguesa
%\usepackage[T1]{fontenc}        % Permite digitar os acentos de forma normal
\usepackage[utf8]{inputenc}
%\usepackage[english]{babel}
% \usepackage[portuges,brazil]{babel}
\usepackage[brazilian]{babel}

%\usepackage[latin1]{inputenc}
\usepackage[dvips]{graphicx}    % Permite Gráficos
%\usepackage{times}    % Fonte Times
\usepackage{fancyhdr}
\usepackage{array}
\usepackage{multicol}
\usepackage[colorlinks=true,linkcolor=blue,urlcolor=blue]{hyperref}
\usepackage{nomencl}    % glossario
\usepackage{amssymb}
\usepackage{amsmath}
\usepackage[compact]{titlesec}
\usepackage{wrapfig}
\usepackage{color}
\usepackage{listingsutf8}

% Configuração do estilo de código c++ do listingsutf8
\definecolor{clr-background}{RGB}{255,255,255}
\definecolor{clr-text}{RGB}{0,0,0}
\definecolor{clr-string}{RGB}{163,21,21}
\definecolor{clr-namespace}{RGB}{0,0,0}
\definecolor{clr-preprocessor}{RGB}{128,128,128}
\definecolor{clr-keyword}{RGB}{0,0,255}
\definecolor{clr-type}{RGB}{43,145,175}
\definecolor{clr-variable}{RGB}{0,0,0}
\definecolor{clr-constant}{RGB}{111,0,138} % macro color
\definecolor{clr-comment}{RGB}{0,128,0}
\definecolor{clr-linenumber}{RGB}{128,128,128}

\lstdefinestyle{cppStyle}{
    language=C++,
    backgroundcolor=\color{clr-background},
    basicstyle=\ttfamily\footnotesize\color{clr-text}, % any text
    stringstyle=\color{clr-string},
    identifierstyle=\color{clr-variable}, % just about anything that isn't a directive, comment, string or known type
    commentstyle=\color{clr-comment},
    directivestyle=\color{clr-preprocessor}, % preprocessor commands
    % listings doesn't differentiate between types and keywords (e.g. int vs return)
    keywordstyle=\color{clr-type},
    keywordstyle={[2]\color{clr-constant}}, % you'll need to define these or use a custom language
    numberstyle=\tiny\color{clr-linenumber},
    %numbers=left,
    showspaces=false,
    showtabs=false,
    showstringspaces=false,
    breaklines=true,
    tabsize=2,
    texcl=true
}

%=======================================================================

% Hifenização das palavras desconhecidas pelo LaTeX
%\hyphenation{}
\paperheight    297mm
\paperwidth     210mm
\voffset         -15mm
\headheight      15pt %% tamanho de letra
\headsep         5mm  %% para o início do texto
\oddsidemargin  -3.0mm
\evensidemargin -3.0mm
\textwidth      167.0mm
\topmargin      005.0mm
\textheight     240.0mm
\footskip       10.0mm

\title{SAET 2024 - Maratona de Programação}

\author{Maratona de Programação}
\date{29 de novembro de 2024}
\usepackage{indentfirst}
\usepackage{subfig}

\parindent=0pt
\setlength{\parskip}{7pt plus 1pt minus 2pt}
\titlespacing{\section}{0pt}{*0}{*0}
\titlespacing{\subsection}{0pt}{*0}{*0}
\titlespacing{\subsubsection}{0pt}{*0}{*0}

\begin{document}

\begin{center}
\textbf{\Huge SAET 2024 - Maratona de Programação} \\
\vspace{0.2cm}
\textit{29 de novembro de 2024} \\
\vspace{1.0cm}
%\textbf{Sevidor BOCA:} \\
%\texttt{\large http://maratona.c3sl.ufpr.br/boca/} \\
%\vspace{1.0cm}
\begin{figure}[h!]
	\centering
 \includegraphics[scale=0.95]{capa.png}
\end{figure}
\vspace{1.0cm}
%\textbf{Organizadores:}\\
%{\small Flávio Zavan} \\
%{\small Ricardo Oliveira} \\
\vspace{1.0cm}
\end{center}

\clearpage

\pagestyle{fancy}
\renewcommand{\footrulewidth}{0.7pt}
\renewcommand{\headrulewidth}{0.7pt}
\lhead{SAET 2024}
\chead{Maratonas de Programação}
\rhead{29 de novembro de 2024}
\cfoot{\thepage}

%\newpage
%\section*{A:} %tle=1
%A ideia principal deste problema é encontrar o número de duplicatas dentro de um intervalo de magias em que a compra seja possível, para isto vamos utilizar ``two pointers" (dois ponteiros) no qual um marca o começo do intervalo e outro marca o final do intervalo, vamos chamá-los $P_l$ e $P_r$.

$P_l$ e $P_r$ iniciam no $i_0$, iremos avançar $P_r$ até $i_{n-1}$. Caso o intervalo se torne maior do que $K$ iremos diminuir o intervalo avançando $P_l$. Devemos atualizar o número de duplicatas ao fazer qualquer avanço em $P_r$ ou $P_l$.

Podemos utilizar da estrutura map para guardar a quantidade de vezes que um número apareceu dentro de nosso intervalo e assim determinar o número de pares em cada avanço.

Complexidade esperada: $\mathcal{O}(N)$.

%\lstinputlisting[title=\large\textbf{.cpp}, style=cppStyle]{}

%\newpage
%\section*{B:} %tle=1
%A ideia principal deste problema é encontrar o número de duplicatas dentro de um intervalo de magias em que a compra seja possível, para isto vamos utilizar ``two pointers" (dois ponteiros) no qual um marca o começo do intervalo e outro marca o final do intervalo, vamos chamá-los $P_l$ e $P_r$.

$P_l$ e $P_r$ iniciam no $i_0$, iremos avançar $P_r$ até $i_{n-1}$. Caso o intervalo se torne maior do que $K$ iremos diminuir o intervalo avançando $P_l$. Devemos atualizar o número de duplicatas ao fazer qualquer avanço em $P_r$ ou $P_l$.

Podemos utilizar da estrutura map para guardar a quantidade de vezes que um número apareceu dentro de nosso intervalo e assim determinar o número de pares em cada avanço.

Complexidade esperada: $\mathcal{O}(N)$.

%\lstinputlisting[title=\large\textbf{.cpp}, style=cppStyle]{}

%\newpage
%\section*{C:} %tle=1
%A ideia principal deste problema é encontrar o número de duplicatas dentro de um intervalo de magias em que a compra seja possível, para isto vamos utilizar ``two pointers" (dois ponteiros) no qual um marca o começo do intervalo e outro marca o final do intervalo, vamos chamá-los $P_l$ e $P_r$.

$P_l$ e $P_r$ iniciam no $i_0$, iremos avançar $P_r$ até $i_{n-1}$. Caso o intervalo se torne maior do que $K$ iremos diminuir o intervalo avançando $P_l$. Devemos atualizar o número de duplicatas ao fazer qualquer avanço em $P_r$ ou $P_l$.

Podemos utilizar da estrutura map para guardar a quantidade de vezes que um número apareceu dentro de nosso intervalo e assim determinar o número de pares em cada avanço.

Complexidade esperada: $\mathcal{O}(N)$.

%\lstinputlisting[title=\large\textbf{}, style=cppStyle]{}

%\newpage
%\section*{D:} %tle=1
%A ideia principal deste problema é encontrar o número de duplicatas dentro de um intervalo de magias em que a compra seja possível, para isto vamos utilizar ``two pointers" (dois ponteiros) no qual um marca o começo do intervalo e outro marca o final do intervalo, vamos chamá-los $P_l$ e $P_r$.

$P_l$ e $P_r$ iniciam no $i_0$, iremos avançar $P_r$ até $i_{n-1}$. Caso o intervalo se torne maior do que $K$ iremos diminuir o intervalo avançando $P_l$. Devemos atualizar o número de duplicatas ao fazer qualquer avanço em $P_r$ ou $P_l$.

Podemos utilizar da estrutura map para guardar a quantidade de vezes que um número apareceu dentro de nosso intervalo e assim determinar o número de pares em cada avanço.

Complexidade esperada: $\mathcal{O}(N)$.

%\lstinputlisting[title=\large\textbf{}, style=cppStyle]{}

%\newpage
%\section*{E:} %tle=1
%A ideia principal deste problema é encontrar o número de duplicatas dentro de um intervalo de magias em que a compra seja possível, para isto vamos utilizar ``two pointers" (dois ponteiros) no qual um marca o começo do intervalo e outro marca o final do intervalo, vamos chamá-los $P_l$ e $P_r$.

$P_l$ e $P_r$ iniciam no $i_0$, iremos avançar $P_r$ até $i_{n-1}$. Caso o intervalo se torne maior do que $K$ iremos diminuir o intervalo avançando $P_l$. Devemos atualizar o número de duplicatas ao fazer qualquer avanço em $P_r$ ou $P_l$.

Podemos utilizar da estrutura map para guardar a quantidade de vezes que um número apareceu dentro de nosso intervalo e assim determinar o número de pares em cada avanço.

Complexidade esperada: $\mathcal{O}(N)$.

%\lstinputlisting[title=\large\textbf{}, style=cppStyle]{}

%\newpage
%\section*{F:} %tle=1
%A ideia principal deste problema é encontrar o número de duplicatas dentro de um intervalo de magias em que a compra seja possível, para isto vamos utilizar ``two pointers" (dois ponteiros) no qual um marca o começo do intervalo e outro marca o final do intervalo, vamos chamá-los $P_l$ e $P_r$.

$P_l$ e $P_r$ iniciam no $i_0$, iremos avançar $P_r$ até $i_{n-1}$. Caso o intervalo se torne maior do que $K$ iremos diminuir o intervalo avançando $P_l$. Devemos atualizar o número de duplicatas ao fazer qualquer avanço em $P_r$ ou $P_l$.

Podemos utilizar da estrutura map para guardar a quantidade de vezes que um número apareceu dentro de nosso intervalo e assim determinar o número de pares em cada avanço.

Complexidade esperada: $\mathcal{O}(N)$.

%\lstinputlisting[title=\large\textbf{}, style=cppStyle]{}

%\newpage
%\section*{G:} %tle=1
%A ideia principal deste problema é encontrar o número de duplicatas dentro de um intervalo de magias em que a compra seja possível, para isto vamos utilizar ``two pointers" (dois ponteiros) no qual um marca o começo do intervalo e outro marca o final do intervalo, vamos chamá-los $P_l$ e $P_r$.

$P_l$ e $P_r$ iniciam no $i_0$, iremos avançar $P_r$ até $i_{n-1}$. Caso o intervalo se torne maior do que $K$ iremos diminuir o intervalo avançando $P_l$. Devemos atualizar o número de duplicatas ao fazer qualquer avanço em $P_r$ ou $P_l$.

Podemos utilizar da estrutura map para guardar a quantidade de vezes que um número apareceu dentro de nosso intervalo e assim determinar o número de pares em cada avanço.

Complexidade esperada: $\mathcal{O}(N)$.

%\lstinputlisting[title=\large\textbf{}, style=cppStyle]{}

%\newpage
%\section*{H:} %tle=1
%A ideia principal deste problema é encontrar o número de duplicatas dentro de um intervalo de magias em que a compra seja possível, para isto vamos utilizar ``two pointers" (dois ponteiros) no qual um marca o começo do intervalo e outro marca o final do intervalo, vamos chamá-los $P_l$ e $P_r$.

$P_l$ e $P_r$ iniciam no $i_0$, iremos avançar $P_r$ até $i_{n-1}$. Caso o intervalo se torne maior do que $K$ iremos diminuir o intervalo avançando $P_l$. Devemos atualizar o número de duplicatas ao fazer qualquer avanço em $P_r$ ou $P_l$.

Podemos utilizar da estrutura map para guardar a quantidade de vezes que um número apareceu dentro de nosso intervalo e assim determinar o número de pares em cada avanço.

Complexidade esperada: $\mathcal{O}(N)$.

%\lstinputlisting[title=\large\textbf{}, style=cppStyle]{}

%\newpage
%\section*{I:} %tle=1
%A ideia principal deste problema é encontrar o número de duplicatas dentro de um intervalo de magias em que a compra seja possível, para isto vamos utilizar ``two pointers" (dois ponteiros) no qual um marca o começo do intervalo e outro marca o final do intervalo, vamos chamá-los $P_l$ e $P_r$.

$P_l$ e $P_r$ iniciam no $i_0$, iremos avançar $P_r$ até $i_{n-1}$. Caso o intervalo se torne maior do que $K$ iremos diminuir o intervalo avançando $P_l$. Devemos atualizar o número de duplicatas ao fazer qualquer avanço em $P_r$ ou $P_l$.

Podemos utilizar da estrutura map para guardar a quantidade de vezes que um número apareceu dentro de nosso intervalo e assim determinar o número de pares em cada avanço.

Complexidade esperada: $\mathcal{O}(N)$.

%\lstinputlisting[title=\large\textbf{}, style=cppStyle]{}

%\newpage
%\section*{J:} %tle=1
%A ideia principal deste problema é encontrar o número de duplicatas dentro de um intervalo de magias em que a compra seja possível, para isto vamos utilizar ``two pointers" (dois ponteiros) no qual um marca o começo do intervalo e outro marca o final do intervalo, vamos chamá-los $P_l$ e $P_r$.

$P_l$ e $P_r$ iniciam no $i_0$, iremos avançar $P_r$ até $i_{n-1}$. Caso o intervalo se torne maior do que $K$ iremos diminuir o intervalo avançando $P_l$. Devemos atualizar o número de duplicatas ao fazer qualquer avanço em $P_r$ ou $P_l$.

Podemos utilizar da estrutura map para guardar a quantidade de vezes que um número apareceu dentro de nosso intervalo e assim determinar o número de pares em cada avanço.

Complexidade esperada: $\mathcal{O}(N)$.

%\lstinputlisting[title=\large\textbf{}, style=cppStyle]{}

\newpage
\section*{K: Kummirub: 2D++!} %tle=1
A ideia principal deste problema é encontrar o número de duplicatas dentro de um intervalo de magias em que a compra seja possível, para isto vamos utilizar ``two pointers" (dois ponteiros) no qual um marca o começo do intervalo e outro marca o final do intervalo, vamos chamá-los $P_l$ e $P_r$.

$P_l$ e $P_r$ iniciam no $i_0$, iremos avançar $P_r$ até $i_{n-1}$. Caso o intervalo se torne maior do que $K$ iremos diminuir o intervalo avançando $P_l$. Devemos atualizar o número de duplicatas ao fazer qualquer avanço em $P_r$ ou $P_l$.

Podemos utilizar da estrutura map para guardar a quantidade de vezes que um número apareceu dentro de nosso intervalo e assim determinar o número de pares em cada avanço.

Complexidade esperada: $\mathcal{O}(N)$.

\lstinputlisting[title=\large\textbf{kummirub.cpp}, style=cppStyle]{kummirub/kummirub2.cpp}

%\newpage
%\section*{L:} %tle=1
%A ideia principal deste problema é encontrar o número de duplicatas dentro de um intervalo de magias em que a compra seja possível, para isto vamos utilizar ``two pointers" (dois ponteiros) no qual um marca o começo do intervalo e outro marca o final do intervalo, vamos chamá-los $P_l$ e $P_r$.

$P_l$ e $P_r$ iniciam no $i_0$, iremos avançar $P_r$ até $i_{n-1}$. Caso o intervalo se torne maior do que $K$ iremos diminuir o intervalo avançando $P_l$. Devemos atualizar o número de duplicatas ao fazer qualquer avanço em $P_r$ ou $P_l$.

Podemos utilizar da estrutura map para guardar a quantidade de vezes que um número apareceu dentro de nosso intervalo e assim determinar o número de pares em cada avanço.

Complexidade esperada: $\mathcal{O}(N)$.

%\lstinputlisting[title=\large\textbf{}, style=cppStyle]{}

\end{document}
