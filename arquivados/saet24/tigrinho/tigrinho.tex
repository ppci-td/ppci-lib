Ricardinho e seu avô Gepeto estão passeando em um domingo à noite pelo circo até que se deparam com um jogo muito interessante, 
o Jogo do Tigrinho, que consiste em a partir de dois textos escritos por diferentes pessoas, determinar se um terceiro texto foi escrito ou não por uma daquelas pessoas.
De fato suas regras são bem simples:
\begin{itemize}
    \item O participante irá ler os dois primeiros textos, um de cada autor.
    \item Em seguida lê o terceiro texto, desta vez com autor desconhecido.
    \item Por fim, infere quem o escreveu sendo \textbf{1} para o primeiro, \textbf{2} para o segundo autor e \textbf{0} caso não foi escrito por nenhum deles.
    \item Os textos não precisam ter algum significado real, são apenas palavras ou sequências de letras espaçadas.
    \item Cada palavra que aparece nos textos se trata de um evento independente.
\end{itemize}

Seu papel é ajudar o Gepeto, um craque da probabilide (visto que determinada palavra tem diferentes probabilidades de 
aparição para cada autor, baseado nas evidencias fornecidas), a desvendar de quem é o texto escrito.

\subsection*{Entrada}

A primeira linha contém um número $a$ $(10 \leq a \leq 10^6)$, indicando o tamanho do texto do primeiro autor. 
Na próxima linha uma sequência de $a$ palavras com tamanho $l$ $(1 \leq l \leq 15)$ são apresentadas, correspondentes ao texto 1.

A terceira linha contém um número $b$ $(10 \leq b \leq 10^6)$, indicando o tamanho do texto do segundo autor.
Na próxima linha uma sequência de $b$ palavras com tamanho $l$ $(1 \leq l \leq 15)$ são apresentadas, correspondentes ao texto 2.

A quinta linha contém um número $n$, que indica o tamanho do texto com autor desconhecido.
E na próxima linha, $n$ palavras correspondente a este texto são apresentadas.

\subsection*{Saída}
A saída deve ser, de acordo com as probabilidades:
\begin{itemize}
    \item $1$ caso a maior probabilidade seja do primeiro autor.
    \item $2$ caso a maior probabilidade seja do segundo autor.
    \item $0$ caso a probabilidade seja 0 para ambos.
\end{itemize}
%----- Exemplo 1 -----%
\newpage
\begin{table}[!h]
\centering
\begin{tabular}{|p{4in}|p{2in}|}
\hline
\begin{minipage}[t]{3in}
\raggedright
\textbf{Exemplo de entrada} \\
\begin{verbatim}
20
um dos personagens da disney que aparece 
no filme do shrek e o pinoquio junto de 
outras famosas princesas ficticias
20
voce sabe quem foi gepeto ele e um 
personagem ficticio italiano entalhador de 
madeira idoso e pobre criador do pinoquio
7
o pinoquio foi criado pelo italiano gepeto
\end{verbatim}
\vspace{1mm}
\end{minipage}
&
\begin{minipage}[t]{3in}
\raggedright
\textbf{Exemplo de saída} \\
\begin{verbatim}
2
\end{verbatim}
\vspace{1mm}
\end{minipage}
\\ \hline
\end{tabular}
\end{table}

\begin{table}[!h]
\centering
\begin{tabular}{|p{4in}|p{2in}|}
\hline
\begin{minipage}[t]{3in}
\raggedright
\textbf{Exemplo de entrada} \\
\begin{verbatim}
86
no futebol os jogadores correm pelo campo
driblando adversarios enquanto passam a 
bola de pe em pe o objetivo principal e 
marcar um gol acertando a bola na rede 
alem da trave a preparacao fisica e mental 
e essencial para garantir bom desempenho 
o jogo e rapido exigindo precisao e 
trabalho em equipe a grama do campo e o 
palco onde a acao acontece e a vitoria e 
o sonho de todos os envolvidos cada 
segundo conta e a tecnica de controle 
da bola e crucial
81
na natacao os atletas deslizam pela agua 
avancando rapidamente pela raia a tecnica 
precisa aliada a forca e a resistencia e 
o que garante o sucesso assim como no 
futebol a preparação fisica e mental e 
indispensavel cada bracada na piscina deve 
ser calculada para maximizar a velocidade 
o tempo e um fator determinante e o 
objetivo e sempre cruzar a linha de chegada 
primeiro o ambiente de competicao como no 
campo de futebol exige foco e concentracao 
totais  dos nadadores
11
a preparacao fisica e mental e crucial 
para vencer no futebol
\end{verbatim}
\vspace{1mm}
\end{minipage}
&
\begin{minipage}[t]{3in}
\raggedright
\textbf{Exemplo de saída} \\
\begin{verbatim}
1
\end{verbatim}
\vspace{1mm}
\end{minipage}
\\ \hline
\end{tabular}
\end{table}
\newpage
