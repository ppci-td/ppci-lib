Era uma vez um jovem programador chamado Lucas, que vivia em um vilarejo apaixonado por desafios de lógica. Certo dia, Lucas recebeu uma mensagem misteriosa que lhe propunha um enigma muito antigo e fascinante, conhecido como Sequência de Fibonacci.

A mensagem dizia:

"A Sequência de Fibonacci é formada de tal forma que cada número, a partir do terceiro, é a soma dos dois anteriores. Ela começa assim: 0, 1, 1, 2, 3, 5, 8, 13... e continua indefinidamente. Sua missão, se aceitar, é criar um algoritmo que encontre o enésimo termo dessa sequência."

O desafio era claro: dado um número n, Lucas deveria encontrar o n-ésimo termo da sequência. Ele sabia que Fibonacci tinha diversas aplicações, desde a natureza até a ciência da computação, e não podia deixar de aceitar esse enigma intrigante.

Sua missão: Construa um algoritmo que, dado um número n, retorne o enésimo número da sequência de Fibonacci.

Lucas então se sentou em sua mesa de trabalho, pronto para decifrar o mistério. Será que você conseguiria ajudá-lo a resolver esse enigma?

\subsection*{Entrada}

A entrada deverá ser um número inteiro correspondente a o n-ésimo termo.

\subsection*{Saída}

Imprima uma única linha com o valor do n-ésimo termo.

%----- Exemplo 1 -----%
\begin{table}[!h]
\centering
\begin{tabular}{|l|l|}
\hline
\begin{minipage}[t]{3in}
\textbf{Exemplo de entrada}
\begin{verbatim}
25
\end{verbatim}
\vspace{1mm}
\end{minipage}
&
\begin{minipage}[t]{3in}
\textbf{Exemplo de saída}
\begin{verbatim}
46368
\end{verbatim}
\vspace{1mm}
\end{minipage} \\
\hline
\end{tabular}
\end{table}

%----- Exemplo 2 -----%
\begin{table}[!h]
\centering
\begin{tabular}{|l|l|}
\hline
\begin{minipage}[t]{3in}
\textbf{Exemplo de entrada}
\begin{verbatim}
5
\end{verbatim}
\vspace{1mm}
\end{minipage}
&
\begin{minipage}[t]{3in}
\textbf{Exemplo de saída}
\begin{verbatim}
3
\end{verbatim}
\vspace{1mm}
\end{minipage} \\
\hline
\end{tabular}
\end{table}
